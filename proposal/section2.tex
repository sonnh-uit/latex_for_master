\section{Giới thiệu}
\label{introduction}

Phần này nói chi tiết hơn nhưng cũng là giới thiệu đề tài. Phần này thông thường bao gồm

\begin{itemize}
    \item Trình bày các thông tin cơ bản, các nghiên cứu nổi tiếng trong lĩnh vực, các vấn đề được đặt ra và các kết quả của nó
    \item Trình bày cụ thể một vài nghiên cứu lien quan. Họ làm gì, làm như thế nào, kết quả là gì. Ưu và nhược điểm của họ. Thông thường sẽ nêu một số nhược điểm mà nghiên cứu của chúng ta sẽ khắc phục.
    \item Trình bày nghiên cứu mới của mình. Mục đích của nghiên cứu là gì, tại sao nó quan trọng và cần phải nghiên cứu. Liệt kê các câu hỏi và giả thuyết nghiên cứu. Phát hiện hoăc nghiên cứu của bạn là gì, giá trị của nó ra sao.
    \item Nêu lên cấu trúc bài nghiên cứu
\end{itemize}

Trong trường hợp kết quá nghiên cứu là một công cụ hoàn chỉnh, phần này cũng nê nói rõ đầu vào, đầu ra của ứng dụng là gì. Nếu là học máy, nêu khái quát về bộ dữ liệu sử dụng