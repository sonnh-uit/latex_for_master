\section{Giới thiệu}

Sự ra đời của Big Data và Machine Learning đã thúc đẩy nhiều nhà khoa học và các tập đoàn tập trung sang phát triển lĩnh vực này, khiến nhu cầu về lưu trữ dữ liệu tăng nhanh. Do đó, các giải pháp quản lý và lưu trữ dữ liệu mới được phát triển. Trong nghiên cứu này, tác giả giới thiệu và thực sự một số đánh giá hiệu năng của HDFS\cite{borthakur2007hadoop}, một phần mềm cung cấp hệ thống lưu trữ phân tán được thiết kế với khả năng dễ mở rộng, có hiệu quả cao, khả năng chịu lỗi tốt, đồng thời có chi phí triển khai thấp. 

Sự phức tạp của hệ phân tán nói chung và HDFS nói riêng đòi hỏi các cấu hình và điều chỉnh cẩn thận để đạt hiệu suất tốt, các thay đổi được thực hiện ảnh hưởng tới toàn hệ thống và có khả năng làm mất mát dữ liệu nếu không được thao tác đúng. Do đó, tìm hiểu và đánh giá hiệu suất toàn diện của hệ thống là cần thiết. Trong nghiên cứu này, tác giả xây dựng và trả lời các câu hỏi sẽ giúp đáp ứng một số nhu cầu lưu trữ dữ liệu phân tán lớn, yêu cầu khả năng chịu lỗi cao, thuận tiện trong phân quyền và đồng thời tốc độ truy cập nhanh. Khả năng tích hợp với các ứng dụng khác cũng được khảo sát và đề xuất. Trong trường hợp nào thì hệ thống hoạt động tốt nhất? Nếu như vậy thì cái giá phải đánh đổi là gì? Hệ thống hoạt động thế nào khi dữ liệu tăng lên cao hoặc nhiều người dùng cùng truy cập? Khi xảy ra hư hỏng phần cứng thì ứng dụng sẽ điều hướng lưu trữ như thế nào? Các giải pháp theo dõi hiệu suất hệ thống khả dụng là gì? Khi nào thì hiệu suất của hệ thống giảm?

Phần còn lại của nghiên cứu được trình bày như sau: Phần \ref{section2-background} trình bày các thành phần cơ bản của HDFS và các nghiên cứu liên quan. Phần \ref{section3-method} trình bày thiết kế hệ thống và khác thực nghiệm. Phần \ref{section4-experiment} trình bày chi tiết các kết quả. Cuối cùng, kết luận và đề xuất được trình bày ở phần \ref{section5-conclusion}.